%%-------------------------------------
\documentclass[12pt]{article} % Document class article only supports 10pt, 11pt, and 12pt.
\usepackage[utf8]{inputenc}
\usepackage[english]{babel}
\usepackage{cx}
\usepackage{tcolorbox}
\usepackage{geometry}
\geometry{left=3cm,right=3cm,top=2cm,bottom=2cm} %% 设置页边距:

\usepackage{indentfirst}

%% 行距、段距设置
%\linespread{1.2}     % 设置基本行距。
% article 文档类的默认是1,即1.2倍字号大小;
% ctexart 文档类的默认是1.3,即1.56倍字号大小。
\setlength\parskip{5pt}   % 段间距

\usepackage[
	backend=bibtex,
	style=numeric-comp,
	sorting=nyt,
	date=year,
	backref=true,
	backrefstyle=three]
	{biblatex}
\addbibresource{/Users/chenxin/Desktop/chenxin-biblatex.bib}




\newcommand{\monic}{\rightarrowtail}
\newcommand{\epic}{\twoheadrightarrow}
\newcommand{\monoepiarrow}{\rightarrowtail\hspace{-1em}\twoheadrightarrow}
\newcommand{\isoarrow}{\xrightarrow{\sim}}

%%=============================================================================
\title{Notes on Computability \& Complexity}
\author{\textsc{Xin Chen} \qquad \href{mailto:chenxin_hello@outlook.com}{\textsf{chenxin\_hello@outlook.com}}  
	\qquad 
	$Q_{uality} = \int (\text{\Large $K$},P,t)$}
\date{latest update: \today}

%%-------------------------------------
\begin{document}


\maketitle

\begin{quotation}
	\itshape May the force of \textsf{P} and \textsf{NP} be with you. 	
\end{quotation}


\tableofcontents


\medskip

Citation testing: \cite{Aro.Bar2009}

\vspace{2em}

One of the important scientific advances in the first half of the twentieth century was that the notion of ``computation'' received a much more precise definition.


At roughly 50 years (1970s -- 2020s), 
complexity theory is still an infant science, 
and many important results are less than 30 years old.



\section{Basic concepts}



If $S$ is a \textit{finite} set, called \textit{alphabet set}, 
then a \textit{string} over $S$ is a finite ordered tuple of elements from $S$. 

We will typically consider the \textit{binary} alphabet $2 = \{0,1\}$.


$S^0 = \{\epsilon\}$

$S^* = \bigcup_{n \geq 0} S^n$ is the set of all strings over $S$.


The \textit{concatenation} of strings $x,y$ is denoted by $x^\frown y$, $x \circ y$, or simply $xy$.


$x^k$ denotes the concatenation of $k$ copies of $x$ for $k \geq 1$. 
% 
For example, 
$1^3$ is `$111$'.

The length of a string $x$ is denoted by $|x|$.


%%=============================================================================
\subsection{Representations}


we implicitly identify any function $f$ whose domain and range are not strings with the function 
\[
    g \colon \{0,1\}^* \to \{0,1\}^*
\]
that given a representation of an object $x$ as input, 
outputs the representation of $f (x)$. 


%%=============================================================================
\subsection{Big--Oh}






% /////////////////////////////////////////////////////////////////////////////
\clearpage
\printbibliography
\addcontentsline{toc}{section}{References}
\end{document}